SHA1は誤り検出符号の一種で, MD5と同様に認証やデジタル署名などに使用される.
計算方法はMD5と同様に文字列を全てbit列へと変換を行い, bit列の長さを一定にするためにパディングを行う.
次に, 元の文字列のbit数を8 bitの整数値へと変換を行い, その上位バイトを拡張した文字列から順に加算する.
そして, SHA1の計算のために, 使用関数と使用定数というものを使用する.
使用関数は以下の通りに定義する.\\

$\hspace{80pt} f(t;b,c,d) =$ ($b$ $AND$ $c$) $OR$ (($NOT$ $b$) $AND$ $d$)        $ ( 0 <= t <= 19)$ \\
$ \hspace{90pt} f(t;b,c,d) =$ $b$ $XOR$ $c$ $XOR$ $d$                        $(20 <= t <= 39)$ \\
$\hspace{90pt} f(t;b,c,d) =$ ($b$ $AND$ $c$) $OR$ ($b$ $AND$ $d$) $OR$ ($c$ $AND$ $d$)  $(40 <= t <= 59)$ \\
$\hspace{90pt} f(t;b,c,d) =$ $b$ $XOR$ $c$ $XOR$ $d$                        $(60 <= t <= 79)$ \\


$f(t;b,c,d)$は論理関数のシーケンスである. それぞれの$f(t), 0 <= t <= 79 $では、3 つの32 bitワードの$b, c, d$ を処理し, 1つの32 bitワードを出力する.

使用定数は以下のように定義する.表記は全て16 進数とする.\\

$
\hspace{140pt} K(t) = 5A827999  ( 0 <= t <= 19) \\
\hspace{150pt} K(t) = 6ED9EBA1  (20 <= t <= 39) \\
\hspace{150pt} K(t) = 8F1BBCDC  (40 <= t <= 59) \\
\hspace{150pt} K(t) = CA62C1D6  (60 <= t <= 79) \\
$

そして, 計算用バッファとして以下の変数を定義する.表記は全て16 進数とする.\\

$
\hspace{140pt} H0 = 67452301 \\
\hspace{150pt} H1 = EFCDAB89 \\
\hspace{150pt} H2 = 98BADCFE \\
\hspace{150pt} H3 = 10325476 \\
\hspace{150pt} H4 = C3D2E1F0 \\
$

そして $M(1), M(2), ... , M(n)$ の計算を行う.\\
まず, $M(i)$を、16個のワード$W(0), W(1), ... , W(15)$に分割する. \\
次に, 16 から79 までの$t$に対して、以下の計算を行う。 \\

$\hspace{70pt} W(t) =$ $S(W(t-3)$ $XOR$ $W(t-8)$ $XOR$ $W(t-14)$ $XOR$ $W(t-16))$ \\

A, B, C, D, E をそれぞれ、A = H0, B = H1, C = H2, D = H3, E = H4とし, 0 から79 までの$t$に対して、以下の計算を行う。 \\

$\hspace{110pt}TEMP = S^5(A) + f(t;B,C,D) + E + W(t) + K(t); $\\
$\hspace{120pt} E = D; D = C; C = S^30(B); B = A; A = TEMP;$ \\

最後に$HO, H1, H2, H3, H4$をそれぞれ以下の計算を行う.

$\hspace{140pt}H0 = H0 + A \\
\hspace{150pt} H1 = H1 + B \\
\hspace{150pt} H2 = H2 + C \\
\hspace{150pt} H3 = H3 + D \\
\hspace{150pt} H4 = H4 + E \\
$

これを, $H0$から$H4$まで順に並べた文字列がSHA1となる.
