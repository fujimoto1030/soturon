CRCは誤り検出符号の一種であり, データから特定の定数の剰余を検査用の値として用いる.
主にPNGやZIPなどのファイルのデータ検査に使われている.

計算方法は, まず{0,1}を2元体$F_2$と考える.ビット列を2元体を係数とするメッセージ多項式を$P(Ms)$, メッセージのビット長を$n_m$
とすると, bit列と多項式は以下の通りとなる.
\\
$\hspace{120pt}Ms = Ms_{n_m-1}Ms_{n_m-2}...Ms_2Ms_1Ms_0 \\
P(Ms) = \sum_{i=0}^{n_m-1}{Ms_ix^i} = Ms_{n_m-1}x^{n_m-1} + Ms_{n_m-2}x^{n_m-2} + ... + Ms_2x^2 + Ms_1x + Ms_0 \\
\hspace{120pt}Ms_i \in F_2 , i \in {n_m-1, n_m-2, ... ,2, 1, 0} \\
$

次に, 乗算する多項式を生成多項式$P(G)$といい,
生成多項式の次数+1を$n_g$とすると, bit列と生成多項式は以下の通りとなる.
\\
$\hspace{120pt}G = G_{n_g-1}G_{n_g-2}...G_2G_1G_0 \\
\hspace{70pt}P(G) = \sum_{i=0}^{n_g-1}{G_ix^i} = G_{n_g-1}x^{n_g-1} + G_{n_g-2}x^{n_g-2} + ... + G_2x^2 + G_1x + G_0 \\
\hspace{120pt}G_i \in F_2 , i \in {n_g-1, n_g-2, ... ,2, 1, 0} \\
$

最後に, メッセージ多項式に$x^n_g−1$を掛けて生成多項式で除算して余りを求める.
計算は以下の通りであり, 余りを$P(Cs)$、商を$Q$とする。
\\
$\hspace{120pt}P(Ms)x^{n_g-1} = Q P(G) + P(Cs) \\
\hspace{120pt}P(Ms)x^{n_g-1} - P(Cs) = Q P(G) \\
\hspace{100pt} \Rightarrow P(Ms)x^{n_g-1} - P(Cs) = 0 \bmod P(G) \\
$
